 \begin{titlepage}

{\thispagestyle{empty} \setlength{\parskip}{1cm}
\begin{center}
{\LARGE{\textbf{POLITECHNIKA BIAŁOSTOCKA}}}
\end{center}
\setlength{\parskip}{0.5cm}
\begin{center} {\large{\textbf{WYDZIAŁ INFORMATYKI}}}
\end{center}
\begin{center}
{\large{\textbf{KATEDRA OPROGRAMOWANIA}}}
\end{center}
\setlength{\parskip}{2cm}
\begin{center}
{\LARGE{\textsc{Praca dyplomowa inżynierska}}}
\end{center}
\begin{center}
{\large{\textsc{TEMAT: Aplikacja internetowa do obsługi systemu
GDT w technologii Python/Django}}}
\end{center}


\setlength{\parskip}{2cm}
\begin{flushright}
{\large{\textsc{WYKONAWCA: Mateusz Pernal}}}
\end{flushright}
\setlength{\parskip}{1.5cm}
\begin{flushright}
{\large{PODPIS: .................................}}
\end{flushright}
\setlength{\parskip}{2cm}
\begin{flushleft}
{\large{\textsc{PROMOTOR: dr inż. Krzysztof Jurczuk}}}
\end{flushleft}
\begin{center}
{\large{\textsc{\textbf{BIAŁYSTOK  2019 ROK}}}}
\end{center}
}
\end{titlepage}

\begin{titlepage}
%\thispagestyle{empty}
\oddsidemargin -0.3in
\topmargin     -0.5in
\noindent
\begin{tabular}{cccccc}
\multicolumn{6}{c}{\textbf{Karta dyplomowa}}\\\hline

\multicolumn{2}{|c|}{POLITECHNIKA BIAŁOSTOCKA}&\multicolumn{2}{|c|}{}&\multicolumn{2}{|l|}{Nr albumu}\\

\multicolumn{2}{|l|}{}&\multicolumn{2}{|c|}{}&\multicolumn{2}{|l|}{studenta.....}\\
\cline{5-6}

\multicolumn{2}{|l|}{Wydział Informatyki}&\multicolumn{2}{|l|}{Studia stacjonarne}&\multicolumn{2}{|l|}{Rok Akademicki}\\

\multicolumn{2}{|l|}{}&\multicolumn{2}{|l|}{}&\multicolumn{2}{|l|}{2016/2019}\\
\cline{5-6}

\multicolumn{2}{|l|}{Katedra Oprogramowania}&\multicolumn{2}{|l|}{studia I stopnia}&\multicolumn{2}{|l|}{Kierunek studiów}\\

\multicolumn{2}{|l|}{}&\multicolumn{2}{|c|}{}&\multicolumn{2}{|l|}{Informatyka}\\

%\multicolumn{2}{|l|}{}&\multicolumn{2}{|c|}{}&\multicolumn{2}{|l|}{Specjalność Inżynieria}\\

%\multicolumn{2}{|l|}{}&\multicolumn{2}{|c|}{}&\multicolumn{2}{|l|}{Oprogramowania}\\
\hline

\multicolumn{6}{|c|}{}\\
\multicolumn{6}{|c|}{\textbf{imię i nazwisko}}\\

\multicolumn{6}{|c|}{}\\
\multicolumn{6}{|l|}{\textbf{TEMAT PRACY DYPLOMOWEJ:}}\\
\multicolumn{6}{|l|}{..................................... ...........................}\\

\multicolumn{6}{|c|}{}\\
\multicolumn{6}{|l|}{Zakres pracy:}\\

\multicolumn{6}{|l|}{}\\
\multicolumn{6}{|l|}{1. .........................................................................}\\
\multicolumn{6}{|l|}{2. .........................................................................}\\
\multicolumn{6}{|l|}{3. .........................................................................}\\

\multicolumn{6}{|c|}{}\\
\multicolumn{6}{|c|}{}\\
\multicolumn{3}{|c}{\quad\quad\quad\quad\quad ..........................}&\multicolumn{3}{c|}{..........................}\\
\multicolumn{3}{|c}{\quad\quad\quad\quad\quad \footnotesize{Imię i nazwisko promotora - podpis}}&\multicolumn{3}{c|}{\footnotesize{Imię i nazwisko kierownika katedry - podpis}}\\

\hline

\multicolumn{6}{|c|}{}\\
\multicolumn{6}{|c|}{}\\
\multicolumn{2}{|c}{..........................}&\multicolumn{2}{c}{..........................}&\multicolumn{2}{c|}{..........................}\\
\multicolumn{2}{|c}{\footnotesize{Data wydania tematu pracy dyplomowej}}&\multicolumn{2}{c}{\footnotesize{Regulaminowy termin złożenia}}&\multicolumn{2}{c|}{\footnotesize{Data złożenia pracy dyplomowej}}\\
\multicolumn{2}{|c}{\footnotesize{- podpis promotora}}&\multicolumn{2}{c}{\footnotesize{pracy dyplomowej}}&\multicolumn{2}{c|}{\footnotesize{- potwierdzenie dziekanatu}}\\

\hline

\multicolumn{6}{|c|}{}\\
\multicolumn{6}{|c|}{}\\
\multicolumn{3}{|c}{\quad\quad\quad\quad\quad ..........................}&\multicolumn{3}{c|}{..........................}\\
\multicolumn{3}{|c}{\quad\quad\quad\quad\quad \footnotesize{Ocena promotora}}&\multicolumn{3}{c|}{\footnotesize{Podpis promotora}}\\

\hline

\multicolumn{6}{|c|}{}\\
\multicolumn{6}{|c|}{}\\
\multicolumn{2}{|c}{..........................}&\multicolumn{2}{c}{..........................}&\multicolumn{2}{c|}{..........................}\\
\multicolumn{2}{|c}{\footnotesize{Imię i nazwisko recenzenta}}&\multicolumn{2}{c}{\footnotesize{Ocena recenzenta}}&\multicolumn{2}{c|}{\footnotesize{Podpis recenzenta}}\\

\hline

\end{tabular}
\end{titlepage}



%\vfill \eject

%{
%\thispagestyle{empty}
%\setlength{\parskip}{0cm}
%\begin{flushright}
%Białystok, dn. 20.06.2005
%\end{flushright}
%\begin{flushleft}
%Politechnika Białostocka
%\end{flushleft}
%\begin{flushleft}
%Wydział Informatyki
%\end{flushleft}

%\begin{flushleft}
%\textbf{Jan Nowak}
%\end{flushleft}
%\setlength{\parskip}{3cm}
%\begin{center}
%\Large{Oświadczenie}
%\end{center}
%\setlength{\parskip}{0cm}
%Przyjmuję do wiadomości, że Wydział
%Informatyki Politechniki Białostockiej nabywa prawa autorskie
%dotyczące rezultatów i oprogramowania wytworzonego w ramach
%przygotowania pracy magisterskiej (\textit{Jan Nowak, Obrazowanie medyczne,
%dr inż. Michał Ogórek}). Ich publikacja i wykorzystanie wymaga zgody
%Dziekana Wydziału Informatyki Politechniki Białostockiej.
%\setlength{\parskip}{1.5cm}
%\begin{flushright}
%Podpis magistranta
%\end{flushright}
%\setlength{\parskip}{0cm}
%\begin{flushright}
%..........................
%\end{flushright}
%%\begin{tabular}{l}
%{
%\scriptsize{\textit{Podstawa prawna: Ustawa z dnia 4.02.1994 o prawie autorskim i prawach pokrewnych (Dz. U.  z 1994 r Nr  24).
%Art. 1. Przedmiotem  prawa autorskiego jest każdy przejaw działalności twórczej o indywidualnym charakterze, ustalony
%w jakiejkolwiek postaci, niezależnie od wartości, przeznaczenia i sposobu wyrażenia (utwór).
%Art. 2. W szczególności przedmiotem prawa autorskiego są utwory:
%1.  wyrażone słowem, symbolami matematycznymi, znakami graficznymi (literackie, publicystyczne, naukowe, kartograficzne oraz programy komputerowe).
%Art. 3. Utwór jest przedmiotem prawa autorskiego od chwili ustalenia, chociażby miał postać nieukończoną.
%Art. 4. Ochrona przysługuje twórcy niezależnie od spełnienia jakichkolwiek formalności.
%Ustalenie w rozumieniu tej ustawy następuje w chwili opublikowania, złożenia w formie opisanej, lub wygłoszenia publicznego
%(w formie wykładu, referatu, seminarium itp.).}
%}}
%%\end{tabular}
%}
%\vfill \eject
