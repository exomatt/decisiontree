\chapter{Wizja aplikacji}
% Przykładowy obrazek i odniesienie do niego: \ref{rys1_label}

% \begin{figure}[htb]
% \centering
% \includegraphics[width=11cm]{grafika/rys1.eps}
% \caption{Podpis pod rysunkiem, źródło: \cite{inproceedings_1}}
% \label{rys1_label}
% \end{figure}

% Tekst po obrazku \cite{book_1}.

\section{Wymagania funkcjonalne}
Tworzenie aplikacji należało zacząć od nakreślenia zakresu funkcjonalności, które aplikacja będzie udostępniać użytkownikom. Podstawowym zadaniem, które musi spełniać jest możliwość przeprowadzania eksperymentów przy pomocy systemu GDT. Kolejnym ważnym aspektem jest możliwość zarządzania, wyświetlania, udostępniania oraz edycji poszczególnych eksperymentów. Każdy z użytkowników powinien konkretnie widzieć, które z przeprowadzanych przez niego doświadczeń zostały już ukończone, a które jeszcze są w trakcie lub czekają w kolejce do obliczeń. Aplikacja powinna również przede wszystkim w przejrzysty sposób wyświetlać wyniki doświadczenia w postaci wygenerowanego drzewa decyzyjnego i poszczególnych statystyk. Podczas uruchomienia nowego zadania do obliczenia, użytkownikowi zostanie wyświetlony pasek postępu oraz oszacowana długość trwania całego zadania, przy czym w dowolnym momencie będzie mógł anulować polecenie.  Wraz z możliwością tworzenia eksperymentu nie odłącznym elementem będzie funkcjonalność zarządzania plikami wejściowymi oraz wyjściowymi. Dla użytkowników początkujących zostanie przedstawiona opcja tworzenia podstawowych plików konfiguracyjnych, bez wgłębiania się w bardziej zaawansowane parametry eksperymentu. Dostęp do całości funkcjonalności powinien być tylko dostępny dla zarejestrowanych użytkowników. Natomiast możliwość rejestracji oraz logowania będzie ogólnodostępna.

System kont użytkowników powinien wyróżniać różne role, które definiowałyby dostęp do poszczególnych funkcjonalności aplikacji. Zarządzanie tymi uprawnieniami będzie się odbywać po przez panel administratora. Administrator aplikacji dodatkowo może modyfikować oraz usuwać konta użytkowników. Co więcej z interfejsu admina będzie istniała możliwość edycji rekordów z bazy danych oraz edycja uprawnień do poszczególnych eksperymentów. 

Biorą pod uwagę możliwość udostępniania przez użytkownika doświadczenia innemu użytkownikowi, ważnym aspektem będzie umożliwienie ograniczenia części akcji możliwych do wykonania na eksperymencie. Podstawowym uprawnieniem, którego nie da się zablokować będzie możliwość wyświetlenia wyników eksperymentu. Natomiast reszta funkcjonalności możliwych do wykonania na doświadczeniu, takich jak uruchamianie, kopiowanie, edycja, usuwanie czy też pobieranie plików wejściowych lub wyjściowych może zostać ograniczona. Użytkownik posiadający od kogoś udostępniony eksperyment z pewnymi ograniczeniami, może udostępnić go dalej tylko jeśli posiada nadane prawa do udostępniania, przy czym nie może znieść już nadanych wcześniej ograniczeń.

opsi przypadków użycia itd ///  
\section{Wymagania niefunkcjonalne}
asdasd.
\section{Wykorzystane technologie}
asdasd.