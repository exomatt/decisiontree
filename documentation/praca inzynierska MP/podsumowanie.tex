\chapter*{Podsumowanie}
\addcontentsline{toc}{chapter}{Podsumowanie}
Założeniem pracy dyplomowej było stworzenie aplikacji internetowej do obsługi systemu GDT. W~tym celu został wykorzystany język Python wraz z~frameworkiem Django i~Django REST Framework. Stworzona aplikacja umożliwia w łatwy sposób zarządzania eksperymentami i~danymi. Wyniki eksperymentu są przedstawione w formie graficznej oraz tekstowej. Dodatkowo został stworzony specjalny zestaw ról możliwych do przypisania dla użytkownika. Dzięki temu aplikacja może mieć większe grono użytkowników o~różnym stopniu zaawansowania. Zarządzanie uprawnieniami użytkowników jest możliwe poprzez panel administratora. Ponadto administrator ma możliwość edytowania modeli zapisanych w bazie danych na przykład zmiany praw dostępu do akcji eksperymentu. 

Stworzona aplikacja internetowa cechuje się wysoką intuicyjnością korzystania, a~zarazem łatwością zarządzania. Stosując logiczne rozdzielenie poszczególnych elementów aplikacji w~celu zwiększenia stabilności zastosowano konteneryzacje z~użyciem oprogramowania Docker. Cały system został podzielony na pięć instancji tworzących wspólnie całość. Wykorzystanie takiego rozwiązania umożliwia w~dowolnym momencie wymiany kontenerów aplikacji na przykład na ich nowszą wersję. Podział na elementy pozwalana wdrożyć aplikacje na kilka serwerów, gdzie każdy będzie odpowiadała za jedną instancję. 

Rozwiązanie stworzone w ramach pracy dyplomowej zostało wdrożone i~udostępnione użytkownikom pod adresem \enquote{http://decisiontree.pl}. Dostęp do systemu jest możliwy za pośrednictwem dowolnej przeglądarki internetowej. Mechanizm zakładania kont i logowania jest udostępniony dla każdego. Bardziej zaawansowane zagadnienia wymagają uprzedniej autoryzacji. Wraz ze wzrostem zainteresowania istnieje możliwość rozbudowania zaimplementowanego oprogramowania o dodatkowe metody uczenia maszynowego, czy też mechanizm predykcji oraz bardziej zaawansowane drzewa decyzyjne (jak np. modelowe).