\chapter*{Podsumowanie}
\addcontentsline{toc}{chapter}{Podsumowanie}
Założeniem projektu było stworzenie aplikacji internetowej do obsługi sytemu GDT. W tym celu został wykorzystany język Python wraz z frameworkiem Django i Django REST Framework. Rezultat końcowy aplikacji umożliwia w łatwy sposób zarządzania eksperymentami i plikami. Wyniki doświadczeń są przedstawione w formie czytelnej wizualizacji pod postacią grafu. Dodatkowo został stworzony specjalny zestaw ról możliwych do przypisania dla użytkownika. Dzięki temu aplikacja trafi do większego grona odbiorców. Zarządzanie uprawnieniami użytkowników jest możliwe poprzez panel administratora. Ponadto admin ma możliwość edytowania modeli zapisanych w bazie danych na przykład zmiany praw dostępu do akcji eksperymentu. 

Stworzona aplikacja internetowa cechuje się wysoką intuicyjnością korzystania, a zarazem łatwością zarządzania. Stosując logiczne odłączenie poszczególnych elementów aplikacji w celu zwiększenia stabilności zastosowano konteneryzacje z użyciem oprogramowania Docker. Cały system został podzielony na pięć instancji tworzących wspólnie całość. Wykorzystanie takiego rozwiązania umożliwia w dowolnym momencie zamiany ze sobą kontenerów aplikacji na przykład zmieniając ich wersje. Podział na elementy pozwalana wdrożyć aplikacje na kilka serwerów, gdzie każdy będzie odpowiadała za jedną instancję. 

Rozwiązanie stworzone w ramach pracy dyplomowej zostało wdrożone i udostępnione użytkownikom pod adresem \enquote{decisiontree.pl}. Dostęp do systemu jest dostępny za pośrednictwem dowolnej przeglądarki internetowej. Mechanizm zakładania kont i logowania jest dostępny dla każdego. Bardziej zaawansowane zagadnienia wymagają uprzedniej autoryzacji. Aplikacja z dużym sukcesem może świadczyć rozwiązania z dziedziny drzew decyzyjnych. Wraz ze wzrostem zainteresowania istnieje możliwość rozbudowania zaimplementowanego oprogramowania o dodatkowe metody uczenia maszynowego, czy też mechanizm predykcji. Architektura REST wykorzystana podczas tworzenia wystawia API, dzięki któremu inne systemy mogą wykorzystywać część logiki biznesowej w swojej strukturze za pośrednictwem komunikacji po protokole http.