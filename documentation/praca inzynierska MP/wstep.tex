\chapter*{Wprowadzenie}
\addcontentsline{toc}{chapter}{Wprowadzenie}
Proces myślowy człowieka jest w dużej mierze oparty o schemat podejmowania decyzji. Mają one wpływ na jego życie i przyszłość. Wybór najbardziej optymalnego rozwiązania danego problemu wymaga dokładnej analizy dostępnych informacji. Posiadając wystarczająca ilość danych możemy wykorzystać różne algorytmy, które wskażą właściwy wybór. Mechanizm podejmowania decyzji dotyczy nie tylko człowieka, a wszystkiego co znajduje się w jego otoczeniu. 

Gwałtowny rozwój technologi w dzisiejszych czasach prowadzi do produkowania coraz większej ilości informacji. Firmy gromadzą dane i starają się wyciągnąć z nich jak najlepsze wnioski. Poddając analizie tak duży zbiór informacji wymagane jest zastosowanie narzędzi ułatwiających uzyskanie wyników. Powoduje to rozwój algorytmów, mechanizmów i różnych ścieżek obróbki danych w celu osiągnięcia zadowalających rezultatów. W ciągu ostatnich kilku lat entuzjazm związany z tymi technikami wzrósł gwałtownie i zdominował przemysł. Uczenie maszynowe wraz z analizą danych stanowi trzon funkcjonalności w większości rozwiązaniach produkowanych przez firmy. Odpowiada za takie dziedziny, jak rozpoznawanie mowy czy uniezależnienie sprzętu od człowieka pod postacią autonomicznych samochodów i robotów sprzątających. Wszystkie te systemy polegają na podjęciu jak najszybszej decyzji dającej zadowalający rezultat. 

Rozwiązania z dziedziny uczenia maszynowego również są tworzone przez pracowników Politechniki Białostockiej. Autorski system GDT opiera się na generowaniu drzew decyzyjnych dla zbiorów z danymi wejściowymi. Mechanizm odpowiadający za obliczenie rezultatów wykorzystuje w swoim działaniu algorytmy genetyczne. Obsługa systemu przebiega po przez konsolę systemową, co wiąże się ze znajomością chodź podstawowych komend. W celu ułatwienia dostępu do platformy GDT istnieje potrzeba zaprojektowania interfejsu graficznego opartego o technologie webowe.  

\section*{Cel pracy}
Celem pracy jest stworzenie aplikacji webowej umożliwiającej obsługę systemu GDT (\textit{Global Decision trees}) służącego do generowania drzew decyzyjnych. Strona internetowa umożliwi tworzenie oraz zarządzanie zadaniami uruchamianymi przy pomocy systemu. Jedną z ważniejszych cech programu powinno być danie użytkownikowi możliwość ustawienia parametrów konfiguracyjnych przed wystartowaniem zadania. Aplikacja internetowa powinna także udostępniać opcje związane z wyświetleniem drzewa w postaci graficznej oraz przedstawieniem wyników eksperymentu.
\section*{Zakres pracy}
Zakres pracy obejmuje: 

\begin{itemize}
\item Zapoznanie z systemem GDT,
\item Analiza wymagań aplikacji,
\item Projekt i implementacja aplikacji, 
\item Testy oraz wdrożenie aplikacji.
\end{itemize}


\section*{Organizacja pracy}
Praca została podzielona na cztery główne części. Pierwsze dwa rozdziały dotyczą przedstawienia problemu, określeniu wymagań i sposobu stworzenia projektu. W dalszej części pracy zostanie omówiona architektura aplikacji wraz z zastosowanymi rozwiązaniami. Natomiast prezentacja całego projektu i opis testów został przedstawiony w rozdziale 4. 

Rozdział 1 \enquote{Przedstawienie problemu} składa się z opisu mechanizmu tworzenia drzew decyzyjnych. Przedstawia zagadnienia związane z systemem GDT stworzonym przez pracowników Politechniki Białostockiej. Porusza też temat aplikacji podobnych dostępnych w internecie.

Rozdział 2 \enquote{Wizja aplikacji} odnosi się do analizy wymagań funkcjonalnych i niefunkcjonalnych aplikacji internetowej. Zawiera przedstawienie przypadków użycia w postaci diagramu oraz prezentuje schemat rozwiązania. Dodatkowo obejmuje temat technologi użytej w trakcie implementacji aplikacji.

Rozdział 3 \enquote{Architektura rozwiązania} składa się z opisu poszczególnych rozwiązań użytych podczas tworzenia aplikacji. Prezentuje najważniejsze mechanizmy stworzone na potrzeby projektu. Przedstawia schemat bazy danych wraz z krótkim opisem najważniejszych tabel.

Rozdział 4 \enquote{Prezentacja aplikacji oraz testy} zawiera opis poszczególnych widoków aplikacji, ale także przykładowy sposób ich użycia. Ponadto przedstawia wyniki testów obciążeniowych i manualnych przeprowadzonych na studentach Politechniki Białostockiej.