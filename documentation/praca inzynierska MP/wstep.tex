\chapter*{Wprowadzenie}
\addcontentsline{toc}{chapter}{Wprowadzenie}
Proces myślowy człowieka jest w dużej mierze oparty o~pewien schemat podejmowania decyzji. Podejmowane decyzje mają kluczowy wpływ na jego aktualne życie i~przyszłość. Wybór najbardziej optymalnego rozwiązania danego problemu wymaga dokładnej analizy dostępnych informacji. Posiadając wystarczającą ilość danych możemy wykorzystać różne algorytmy, które mogą pomóc podjąć właściwy wybór. Mechanizm podejmowania decyzji bezpośrednio dotyczy nie tylko człowieka, a~wszystkiego co znajduje się w~jego otoczeniu. 

Szybki rozwój technologi w XX i XXI wieku prowadzi do produkowania i~gromadzenia coraz większej ilości informacji. Firmy starają się wyciągnąć ze zgromadzonych danych możliwe jak najlepsze wnioski. Poddając analizie tak duże zbiory informacji wymagane jest zastosowanie narzędzi uproszczających i~przyśpieszających uzyskanie wyników. Prowadzi to do tworzenia algorytmów oraz mechanizmów zarówno obróbki danych, jak i ich analizy w~celu osiągnięcia zadowalających rezultatów. W przeciągu ostatnich kilkunastu lat entuzjazm związany z~technikami komputerowymi wzrósł gwałtownie i~zdominował przemysł. Uczenie maszynowe wraz z~analizą danych stanowi bardzo ważny element rozwiązań produkowanych przez firmy. Wspomaga takie technologie, jak rozpoznawanie mowy, pisma czy też autonomiczne samochody i~roboty sprzątające. Wszystkie te rozwiązania wymagają przetwarzania ogromnych ilości informacji, w jak najkrótszym czasie oraz podjęcie wystarczająco dobrej decyzji.

W pracy tej rozwijany będzie system do uczenia maszynowego GDT (\textit{Global Decision trees})\cite{sgdt_1} tworzony przez pracowników Politechniki Białostockiej. System ten służy do generowania drzew decyzyjnych na podstawie zbioru uczącego. Drzewa generowane są z~wykorzystaniem algorytmów ewolucyjnych (metoda alternatywna do algorytmów zachłannych typu \textit{top-down}). Aplikacja GDT jest programem konsolowym. W celu ułatwienia dostępu do platformy GDT większemu gronu użytkowników w niniejszej pracy zostanie zaprojektowana, zaimplementowana oraz wdrożona aplikacja do obsługi systemu GDT z~poziomu przeglądarki internetowej.  

%Rozwiązania z dziedziny uczenia maszynowego również są tworzone przez pracowników Politechniki Białostockiej. Autorski system GDT opiera się na generowaniu drzew decyzyjnych dla zbiorów z danymi wejściowymi. Mechanizm odpowiadający za obliczenie rezultatów wykorzystuje w swoim działaniu algorytmy genetyczne. Obsługa systemu przebiega po przez konsolę systemową, co wiąże się ze znajomością chodź podstawowych komend. W celu ułatwienia dostępu do platformy GDT istnieje potrzeba zaprojektowania interfejsu graficznego opartego o technologie webowe.  

\section*{Cel pracy}
Celem pracy jest stworzenie aplikacji webowej umożliwiającej obsługę systemu GDT. Aplikacja ta będzie umożliwiać tworzenie, zlecanie oraz zarządzanie zadaniami uruchamianymi przy pomocy systemu. Podczas tworzenia zadań użytkownik powinien móc ustawić opcje dotyczące, np. wybranego algorytmu oraz jego parametrów. Aplikacja powinna także udostępniać opcje związane z wyświetleniem drzewa wynikowego w postaci graficznej, jego eksport do pliku oraz wgląd do pozostałych wyników uruchomianego algorytmu.
\section*{Zakres pracy}
Zakres pracy obejmuje: 

\begin{itemize}
\item Zapoznanie z systemem GDT,
\item Analiza wymagań aplikacji,
\item Projekt i implementacja aplikacji, 
\item Testy oraz wdrożenie aplikacji.
\end{itemize}


\section*{Organizacja pracy}
Praca została podzielona na cztery główne części. Pierwsze dwa rozdziały zawierają przedstawienie problemu, analizę wymagań i~wykorzystane technologie. W dalszej części pracy została omówiona architektura aplikacji wraz z zastosowanymi rozwiązaniami. Natomiast prezentacja stworzonej aplikacji oraz opis testów został przedstawiony w~rozdziale 4.

Rozdział 1 zawiera opis mechanizmu tworzenia drzew decyzyjnych. Przedstawia również zagadnienia związane z systemem GDT. Porusza też temat podobnych aplikacji dostępnych w~internecie.

Rozdział 2 przedstawia analizę wymagań funkcjonalnych i~niefunkcjonalnych tworzonej aplikacji. Został w nim zamieszczony diagram przypadków użycia wraz z~ich opisem oraz diagram czynności. Następnie zaprezentowany został schemat rozwiązania. Rozdział kończy przedstawienie użytych technologi podczas tworzenia aplikacji.

Rozdział 3 przedstawia architekturę tworzonej aplikacji. Prezentuje najważniejsze mechanizmy oraz rozwiązania stworzone na potrzeby aplikacji. Przedstawia także schemat bazy danych wraz z krótkim opisem najważniejszych tabel.

Rozdział 4 przedstawia stworzoną aplikację. Zawiera on opis poszczególnych widoków aplikacji, ale także przykładowy sposób ich użycia. Ponadto przedstawia wyniki testów obciążeniowych i manualnych przeprowadzonych przez studentów Wydziału Informatyki Politechniki Białostockiej.