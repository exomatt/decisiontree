\chapter*{\centering{\vspace{1in}Summary}}
\addcontentsline{toc}{chapter}{Streszczenie}
 
\epigraphhead[40]{
Subject of diploma thesis: Internet application to support the GDT system in Python/Django technology.}

The aim of this thesis was to design, implement and introduce a web application to support GDT (\textit{Global Decision Trees}) system using Python and Django technologies. In the basic version, GDT is a console program for creating decision trees. The requirement of the project was to create a web application allowing to create, delegate and manage the tasks launched by using the GDT system. The developed tool will also provide graphical representation of the obtained results.

The thesis consists of four chapters. First chapter describes the problem and presents the decision trees. It also shows the most popular existing solutions. Second chapter contains an analysis of project requirements and a overview of the technologies that have been used. Third chapter is intended to present the system architecture. It illustrates most important mechanisms and solutions created for the application. It also shows the database schema with a short description of tables. Fourth chapter contains a overview of particular views of the application, but also an example how to use them. In addition, it presents the results of load and manual tests.
